\documentclass{article}
\usepackage[hyphens,spaces,obeyspaces]{url}
\usepackage{pgfplots}
\usepackage{tikz}
\usepackage{graphicx}
\graphicspath{ {./images/} }
\usepackage{amsmath}
\usepackage{amsthm}
\usepackage{amssymb}
\usepackage{mathabx}
\usepackage{amsfonts}
\usepackage{enumitem}
\graphicspath{ {./images/} }
\usetikzlibrary{shapes}
\usepgfplotslibrary{polar}
\usetikzlibrary{decorations.markings}
\usetikzlibrary{backgrounds}
\pgfplotsset{every axis/.append style={
                    axis x line=middle,    % put the x axis in the middle
                    axis y line=middle,    % put the y axis in the middle
                    axis line style={<->,color=blue}, % arrows on the axis
                    xlabel={$},          % default put x on x-axis
                    ylabel={$},          % default put y on y-axis
            }}
\newcommand{\numpy}{{\tt numpy}}    % tt font for numpy
\usepackage[utf8]{inputenc}

\newtheorem{theorem}{Theorem}
\newtheorem{lemma}[theorem]{Lemma}
\newtheorem{proposition}[theorem]{Proposition}
\newtheorem{corollary}[theorem]{Corollary}
\newtheorem{conjecture}{Conjecture}
\newtheorem{definition}{Definition}
\theoremstyle{remark}
\newtheorem{example}{Example}
\newtheorem{remark}[example]{Remark}

\title{On Determinants}
\author{MinSeok Song}
\date{}
\usepackage{pgfplots}
\pgfplotsset{compat=1.18}
\begin{document}

\maketitle

The goal of this document is to understand determinant of the matrix in a better way.

First, view the matrix of $n\times n$ dimension as a linear transformation. It is helpful to think of determinant as a function that takes 
$n$ many $\mathbb{R}^n$ dimensional vectors in the matrix and outputs a real number. The number it is supposed to tell us is
 the change in oreinted volume by the action of matrix (linear transformation), disregarding the shape of inputs and outputs.

 This heuristics can be made more precise by characterization
  of determinant "function" in the following way.

  \begin{enumerate}
  \item Switching rows change the sign.
  \item It is linear for each element (column) vector.
  \item determinant of the identity matrix is 1.
  \item Multiplying a row by a constant multiplies the determinant by that constnat.
  \end{enumerate}

  This motivates the definition of wedge product in differential topology.\\

We define $$f\wedge g=\frac 1{k!l!}A(f\otimes g)$$ where
$$Af=\sum_{\sigma\in S_k}(\text{sgn } \sigma)\sigma f$$ and $$f\otimes g(v_1,\dots,v_{k+l})=f(v_1,\dots,v_k)g(v_{k+1},\dots,v_{k+l}).$$
  It follows that $$(\alpha^1\wedge\cdots\wedge \alpha^k)(v_1,\dots, v_k)=det[\alpha^i(v_j)]$$

This is exactly an attempt to capture the characteristics of determinant; antilinearity
 corresponds to 1, and linearity corresponds to 2 and 4. Further, 3 corresponds to the normalizing constant $\frac 1{k!l!}$.

 \begin{remark}
 \item It is intuitively clear that under these characterizations $det (AB)=det A*det B$
 \item In this perspective, we can alternatively view the determinant as the product of all singular values, which are simply the degree of stretch in each $n$ direction.
 \item Logarithm translates product to addition, and is a concave in itself. It follows that $\log det(A)$ where $A$ is positive definite
  is concave. We may make this more rigorous by checking $g''(t)\leq 0$ where $g(t)=f(Z+tV)$ for $Z,V\in S^n$.
 
 \end{remark}








\end{document}
