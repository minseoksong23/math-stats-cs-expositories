\documentclass{article}
\usepackage[hyphens,spaces,obeyspaces]{url}
\usepackage{pgfplots}
\usepackage{tikz}
\usepackage{graphicx}
\graphicspath{ {./images/} }
\usepackage{amsmath}
\usepackage{amsthm}
\usepackage{amssymb}
\usepackage{mathabx}
\usepackage{amsfonts}
\usepackage{enumitem}
\usepackage{algorithm}
\usepackage{algorithmic}
\graphicspath{ {./images/} }
\usetikzlibrary{shapes}
\usepgfplotslibrary{polar}
\usetikzlibrary{decorations.markings}
\usetikzlibrary{backgrounds}
\pgfplotsset{every axis/.append style={
                    axis x line=middle,    % put the x axis in the middle
                    axis y line=middle,    % put the y axis in the middle
                    axis line style={<->,color=blue}, % arrows on the axis
                    xlabel={$},          % default put x on x-axis
                    ylabel={$},          % default put y on y-axis
            }}
\newcommand{\numpy}{{\tt numpy}}    % tt font for numpy
\usepackage[utf8]{inputenc}

\newtheorem{theorem}{Theorem}
\newtheorem{lemma}[theorem]{Lemma}
\newtheorem{proposition}[theorem]{Proposition}
\newtheorem{corollary}[theorem]{Corollary}
\newtheorem{conjecture}{Conjecture}
\newtheorem{definition}{Definition}
\theoremstyle{remark}
\newtheorem{example}{Example}
\newtheorem{remark}[example]{Remark}

\title{Distribution, distribution function, random variable and probability measure}
\author{MinSeok Song}
\date{}
\usepackage{pgfplots}
\pgfplotsset{compat=1.18}
\begin{document}
\maketitle
\begin{definition}
    Distribution function is a function $F:\mathbb{R}\to [0,1]$ that is non-decreasing and right-continuous.
    Random variable is a measurable function $X:\Omega\to\mathbb{R}$ where $\Omega$ is equipped with probability measure and $\mathbb{R}$ with Lebesgue measure.
    Probability measure is a measure defined on $\Omega$
    Measure induced by random varialbe is distribution. 
\end{definition}
There are two important theories that look trivial but are not really trivial.
\begin{itemize}
\item The Skorokhod Representation Theorem 
    states that for every CDF, there exists a canonical probability space and a random variable on that space with the given CDF.\\
\item Kolmogorov's Existence threorem states that given finite dimensional sets of distribution, there exists a canonical probability space with random variables whose 
corresponding distributions coincide with our original distributions.
\item The existence of distribution function guarantees the existence of distribution, and vice versa.
\item Thus, Kolmogorov's existence theorem is more general than Skorokhod Representation theorem.
\end{itemize}

\begin{proposition}
    Let us assume that probability measure is complete. Let $X_t=Y_t$ almost everywhere for every $t\geq 0$.
    Then there exists $\tilde{Y}_t=Y_t$ almost everywhere for every $t$ such that $X_t(w)=\tilde{Y}_t(w)$ for every $w\in P$ for every $t$ for 
    measurable set $P$ with measure 1.
\end{proposition}

\end{document}